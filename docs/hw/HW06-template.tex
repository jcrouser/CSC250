% --------------------------------------------------------------
% This is all preamble stuff that you don't have to worry about.
% Head down to where it says "Start here"
% --------------------------------------------------------------
 
\documentclass[12pt]{article}
 
\usepackage[margin=1in]{geometry} 
\usepackage{amsmath,amsthm,amssymb}
\usepackage{graphicx}
\usepackage{enumerate}
\usepackage{xcolor}
\usepackage{tikz}
\definecolor{smithblue}{HTML}{002855}
\definecolor{smithyellow}{HTML}{F2A900}

\usepackage[parfill]{parskip}
\parskip=\baselineskip
 
\newcommand{\N}{\mathbb{N}}
\newcommand{\Z}{\mathbb{Z}}
 
\newenvironment{exercise}[2][Exercise]{\begin{trivlist}
\item[\hskip \labelsep {\bfseries #1}\hskip \labelsep {\bfseries #2.}]}{\end{trivlist}}

\newenvironment{solution}[1][{\color{red} Solution:}]{\begin{trivlist}
\item[\hskip \labelsep {\bfseries #1}\hskip \labelsep {\bfseries}]}{\end{trivlist}}


\usepackage[framemethod=TikZ]{mdframed}
\mdfsetup{%
backgroundcolor=gray!20,
roundcorner=4pt}
\newmdenv[frametitle={\textbf{\textit{Tip:}}}]{tip}


\usepackage{fancyhdr}
\pagestyle{fancy}
\lhead{Submitted by: \studentName\\
\collaborators}
\rhead{CSC250 Spring 2024 - Homework 6\\
\today{}}
\cfoot{p. \thepage}
\renewcommand{\headrulewidth}{0.4pt}
\renewcommand{\footrulewidth}{0.4pt}

\begin{document}
 
% --------------------------------------------------------------
%                         Start here
% --------------------------------------------------------------


\newcommand{\studentName}{YOUR NAME HERE} %replace with your name

\newcommand{\collaborators}{
	% Comment out the line below if you worked alone
	with \textit{COLLABORATORS' NAMES HERE}
	% Uncomment the line below if you worked alone
	% \textit{I did not collaborate with anyone on this assignment.}
}

% --------------
% Exercise 1
% --------------

\begin{tip}
This question is all about practicing reductions! Though it may seem a bit silly to keep re-proving that these languages are undecidable, working with familiar languages enables us to focus on mastering the \textbf{technique} of using reductions.
\end{tip}

\begin{exercise}{1}
Complete the following reductions:

\begin{enumerate}[(a)]
    \item Given that $HALT$ is known to be undecidable, \textbf{prove} that $A_{TM}$ is undecidable by showing that $HALT \leq_T A_{TM}$.
    \item Given that $A_{TM}$ is known to be undecidable, \textbf{prove} that $HALT$ is undecidable by showing that $A_{TM} \leq_T HALT$.
    \item What do parts (a) and (b) tell you about the \textbf{relative difficulty} of $HALT$ and $A_{TM}$?

\end{enumerate}

\end{exercise}

\newpage


% --------------
% Exercise 2
% --------------
\begin{exercise}{2}
In lecture, we proved that a language is Turing-recognizable \textbf{if and only if} it is enumerable. Given a recognizable language $L$ and its corresponding recognizer $R_L$, identify and fix the flaw in the following (faulty) definition of its enumerator.

\noindent\makebox[\textwidth][c]{%
    \begin{minipage}{.5\textwidth}
        \texttt{For each word w in} $\Sigma^*$:\\
\texttt{ \ \ 1. Run} $R_L$ \texttt{on w.}\\
\texttt{ \ \ 2. If it accepts, output w.}
    \end{minipage}}
\end{exercise}


\newpage


% --------------
% Exercise 3
% --------------
\begin{exercise}{3}
Consider the following language: \[MIRROR=\{\langle M\rangle \ | \ M \texttt{ is a TM that accepts } w^R \texttt{ whenever it accepts } w\}\] 
Show that $MIRROR$ is undecidable using a reduction.
\end{exercise}

\newpage

% --------------
% Exercise 4
% --------------
\begin{exercise}{4}
Prove that each of the following languages is undecidable:
\begin{enumerate}[(a)]
\setlength{\itemsep}{10em}
\item $INFINITE_{TM} = \{ \langle M \rangle \; \vert \; M \text{ is a TM and } L(M) \text{ is an infinite language} \} $.

% -------------------------------------------
%  Write your answer to Q3(a) below
% -------------------------------------------

\item $A_{1011} = \{ \langle M \rangle \; \vert \; M \text{ is a TM and } 1011 \in L(M)\} $.

% -------------------------------------------
%  Write your answer to Q3(b) below
% -------------------------------------------

\item $ALL_{TM} = \{ \langle M \rangle \; \vert \; M \text{ is a TM and } L(M) \text{ is }\Sigma^* \} $.

% -------------------------------------------
%  Write your answer to Q3(c) below
% -------------------------------------------

\end{enumerate}
\end{exercise}


% \vskip 2em 
% \hrule
% \vskip 2em 

% -----------------
% References
% -----------------
\vfill
\begin{thebibliography}{9}
 \bibitem{sipser} 
Sipser, Michael. 
\textit{Introduction to the Theory of Computation.}
Course Technology, 2005. ISBN: 9780534950972

\bibitem{critchlow2011foundation}
Critchlow, Carol and Eck, David
\textit{Foundation of computation.},
Critchlow Carol, 2011
\end{thebibliography}


% --------------------------------------------------------------
%     You don't have to mess with anything below this line.
% --------------------------------------------------------------
 
\end{document}