% --------------------------------------------------------------
% This is all preamble stuff that you don't have to worry about.
% Head down to where it says "Start here"
% --------------------------------------------------------------
 
\documentclass[12pt]{article}
 
\usepackage[margin=1in]{geometry} 
\usepackage{amsmath,amsthm,amssymb}
\usepackage{graphicx}
\usepackage{enumerate}
\usepackage{xcolor}
\definecolor{smithblue}{HTML}{002855}
\definecolor{smithyellow}{HTML}{F2A900}

\usepackage[parfill]{parskip}
\parskip=\baselineskip
 
\newcommand{\N}{\mathbb{N}}
\newcommand{\Z}{\mathbb{Z}}
 

\newenvironment{exercise}[2][Exercise]{\begin{trivlist}
\item[\hskip \labelsep {\bfseries #1}\hskip \labelsep {\bfseries #2.}]}{\end{trivlist}}

\newenvironment{solution}[1][{\color{red} Solution:}]{\begin{trivlist}
\item[\hskip \labelsep {\bfseries #1}\hskip \labelsep {\bfseries}]}{\end{trivlist}}


\usepackage{fancyhdr}
\pagestyle{fancy}
\lhead{Submitted by: \studentName\\
\collaborators}
\rhead{CSC250 Spring 2024 - Homework 07\\
\today{}}
\cfoot{p. \thepage}
\renewcommand{\headrulewidth}{0.4pt}
\renewcommand{\footrulewidth}{0.4pt}

\begin{document}
 
% --------------------------------------------------------------
%                         Start here
% --------------------------------------------------------------


\newcommand{\studentName}{YOUR NAME HERE} %replace with your name

\newcommand{\collaborators}{
	% Comment out the line below if you worked alone
	with \textit{COLLABORATORS' NAMES HERE}
	% Uncomment the line below if you worked alone
	% \textit{I did not collaborate with anyone on this assignment.}
}

% --------------
% Exercise 1
% --------------
\begin{exercise}{1}
Consider the following language:
\[COMPOSITE_n = \{n \ | \ n = ab \texttt{ for some integers } a,b\}\]
What is the smallest class that contains this language (finite, regular, context-free, decidable, recognizable, or unrecognizable)? Prove it.
\end{exercise}

% -------------------------------------------
%  Write your answer to Q1 below
% -------------------------------------------
\begin{solution}
Your solution here
\end{solution}

\clearpage

% --------------
% Exercise 2
% --------------
\begin{exercise}{2}
Consider the following language:
\begin{eqnarray*}
COMPOSITE_{TM} & = & \{\langle M,w \rangle \ | \ M \texttt{ is a TM and } M \texttt{ halts on } w \texttt{ in }\\
&& \ \ \ \ \ \ \ \ \ \ \ \ \ n = ab \texttt{ steps for some integers } a,b\}
\end{eqnarray*}
What is the smallest class that contains this language (finite, regular, context-free, decidable, recognizable, or unrecognizable)? Prove it.
\end{exercise}

% -------------------------------------------
%  Write your answer to Q2 below
% -------------------------------------------
\begin{solution}
Your solution here
\end{solution}

\clearpage
% --------------
% Exercise 3
% --------------
\begin{exercise}{3}
Consider the following language:
\[COMPOSITE_{RE} = \{n \ | \ n = ab \texttt{ for some regular expressions } a,b\}\]
What is the smallest class that contains this language (finite, regular, context-free, decidable, recognizable, or unrecognizable)? Prove it.
\end{exercise}

% -------------------------------------------
%  Write your answer to Q3 below
% -------------------------------------------
\begin{solution}
Your solution here
\end{solution}

\clearpage

% --------------
% Exercise 4
% --------------
\begin{exercise}{4}

Describe the primary differences between a Turing reduction ($\le_T$) and a Mapping reduction ($\le_m$).

\end{exercise}

% -------------------------------------------
%  Write your answer to Q4 below
% -------------------------------------------
\begin{solution}
Your solution here
\end{solution}


% -----------------
% References
% -----------------
\vfill
\begin{thebibliography}{9}
\bibitem{sipser} 
Sipser, Michael. 
\textit{Introduction to the Theory of Computation.}
Course Technology, 2005. ISBN: 9780534950972

\bibitem{critchlow2011foundation}
Critchlow, Carol and Eck, David
\textit{Foundation of computation.},
Critchlow Carol, 2011

\end{thebibliography}

% --------------------------------------------------------------
%     You don't have to mess with anything below this line.
% --------------------------------------------------------------
 
\end{document}