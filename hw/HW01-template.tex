% --------------------------------------------------------------
% This is all preamble stuff that you don't have to worry about.
% Head down to where it says "Start here"
% --------------------------------------------------------------
 
\documentclass[12pt]{article}
 
\usepackage[margin=1in]{geometry} 
\usepackage{amsmath,amsthm,amssymb}
\usepackage{graphicx}
\usepackage{enumerate}
\usepackage{xcolor}
\usepackage{lastpage}
\definecolor{smithblue}{HTML}{002855}
\definecolor{smithyellow}{HTML}{F2A900}
\usepackage[framemethod=TikZ]{mdframed}
\mdfsetup{%
backgroundcolor=gray!20,
roundcorner=4pt}
\newmdenv[frametitle={\textbf{\textit{Recommended Solution Format:}}}]{tip}

\usepackage[parfill]{parskip}
\parskip=\baselineskip
 
\newcommand{\N}{\mathbb{N}}
\newcommand{\Z}{\mathbb{Z}}
 
\newenvironment{exercise}[2][Exercise]{\begin{trivlist}
\item[\hskip \labelsep {\bfseries #1}\hskip \labelsep {\bfseries #2.}]}{\end{trivlist}}

\newenvironment{solution}[1][{\color{red} Solution:}]{\begin{trivlist}
\item[\hskip \labelsep {\bfseries #1}\hskip \labelsep {\bfseries}]}{\end{trivlist}}


\usepackage{fancyhdr}
\pagestyle{fancy}
\lhead{Submitted by: \mainName\\
\collaborators}
\rhead{CSC250 Fall 2023 - Homework 01\\
\today{}}
\cfoot{p. \thepage \ of \pageref{LastPage}}
\renewcommand{\headrulewidth}{0.4pt}
\renewcommand{\footrulewidth}{0.4pt}
 

%\newcommand\solution[1]{\vskip 5pt \noindent{\color{red}{\bf Solution:}} \emph{#1}}
 
\begin{document}
 
% --------------------------------------------------------------
%                         Start here
% --------------------------------------------------------------

\newcommand{\mainName}{Your Name Here} %replace with your team name

\newcommand{\collaborators}{
%replace with your member names
	Team: \textit{(Collaborator names here)}
}


% --------------
% Exercise 1
% --------------
\begin{exercise}{1}
For each of the following English arguments, express the argument in terms of \textbf{propositional logic} and determine whether the argument is valid or invalid.

\begin{tip}
\begin{enumerate}
    \item Show the variables you'll use to represent statements.
    \item Fill a table with the truth values of the parts of the propositional logic statements up to and including the stated conclusion.
    \item Add a clear answer of whether or not the stated conclusion is valid.
\end{enumerate}
\end{tip} 

\begin{enumerate}[(a)]

\item If it is sunny and you have finished your homework, you always go for a run. Yesterday, you did not run. I conclude that you did not finish your homework. 
    % -------------------------------------------
    %  Write your answer to Q1a below
    % -------------------------------------------
    \begin{solution} 
        \begin{proof}[\unskip\nopunct]
            Your proof here
        \end{proof}

    
    \end{solution}


    \item Every time you drink coffee after 5 PM, you lose sleep. When you lose sleep, you forget things. You forgot your wallet at home, so I can say for sure that yesterday you had a coffee after 5 PM.
    
    % -------------------------------------------
    %  Write your answer to Q1b below
    % -------------------------------------------
    \begin{solution} 
        \begin{proof}[\unskip\nopunct]
            Your proof here
        \end{proof}
    \end{solution}
    
\end{enumerate}
\end{exercise}

\clearpage

% --------------
% Exercise 2
% --------------
\begin{exercise}{2}

Determine whether each of the following is \texttt{true} or \texttt{false}. 

\begin{tip}
\begin{enumerate}
    \item Indicate whether the statement is \texttt{true} or \texttt{false}.
    \item Add any required proof in concise form (math or a couple of statements). If an example or counterexample are sufficient evidence, use that; if not, provide a general proof.
\end{enumerate}
\end{tip}

\begin{enumerate}[(a)]
	\item Every odd number has an even divisor
	% -------------------------------------------
    %  Write your answer to Q2a below
    % -------------------------------------------
    \begin{solution} 
        \begin{proof}[\unskip\nopunct]
            Your proof here
        \end{proof}
    \end{solution}

    \item The multiplication of two even numbers is even
	% -------------------------------------------
    %  Write your answer to Q2b below
    % -------------------------------------------
    \begin{solution} 
        \begin{proof}[\unskip\nopunct]
            Your proof here
        \end{proof}
    \end{solution}    
        
	\item The multiplication of two odd numbers is odd
	% -------------------------------------------
    %  Write your answer to Q2c below
    % -------------------------------------------
    \begin{solution} 
        \begin{proof}[\unskip\nopunct]
            Your proof here
        \end{proof}
    \end{solution}

    
    
	\end{enumerate}
\end{exercise}

\clearpage

% --------------
% Exercise 3
% --------------
\begin{exercise}{3}

    The \textbf{pigeonhole principle} is the following somewhat intuitive observation: 
    \begin{center}
    If you have $n$ pigeons in $k$ pigeonholes and if $n>k$,\\then there is at least one pigeonhole that contains more than one pigeon.
    \end{center}
    Even though this observation is intuitive, it's always a good idea to verify. Prove the pigeonhole principle using a \textbf{proof by contradiction}.

\begin{tip}
    \begin{enumerate}
        \item Clearly label each of the steps for going from a given statement to a contradiction.
        \item Conclude with what it means to have arrived at that contradiction.\\
        Note: a few sentences are more than enough!
    \end{enumerate}
    \end{tip}
        
\end{exercise}

% -------------------------------------------
%  Write your answer to Q3 below
% -------------------------------------------
\begin{solution} 
        \begin{proof}[\unskip\nopunct]
            Your proof here
        \end{proof}
    \end{solution}

\clearpage

% --------------
% Exercise 4
% --------------

\begin{exercise}{4} Use \textbf{induction} to prove the following statements.

\begin{tip}
    \begin{enumerate}
        \item Label each of the steps
        \item Clearly label each of the mathematical manipulations taken to complete each step.
        \item If needed, add small comments to clarify the meaning of each step.
    \end{enumerate}
\end{tip}

\begin{enumerate}[(a)]
	\item Prove that $1 + 2 + 3 + \dots + n = \frac{n(n+1)}{2}$
	% -------------------------------------------
    %  Write your answer to Q4a below
    % -------------------------------------------
    \begin{solution} 
        \begin{proof}[\unskip\nopunct]
            Your proof here
            \begin{align*} 
            & \text{this is a nice way to align} \\
            & \text{equations, in case} \\
            & \text{you need to}
            \end{align*}
        \end{proof}
    \end{solution}

\item  Prove that $\sum^{n}_{i=1}(2i - 1) = n^2$


	% -------------------------------------------
    %  Write your answer to Q4b below
    % -------------------------------------------
\begin{solution} 
        \begin{proof}[\unskip\nopunct]
            Your proof here
        \end{proof}
    \end{solution}
        
\end{enumerate}        
\end{exercise}

% -----------------
% References
% -----------------
\vfill

\begin{thebibliography}{9}
\bibitem{sipser} 
Sipser, Michael. 
\textit{Introduction to the Theory of Computation.} 
Course Technology, 2005. ISBN: 9780534950972. 

\end{thebibliography}

% --------------------------------------------------------------
%     You don't have to mess with anything below this line.
% --------------------------------------------------------------
 
\end{document}